\documentclass[12pt, a4paper]{article}
\usepackage[utf8]{inputenc}
\usepackage[russian]{babel}
\usepackage[pdftex]{graphicx, color}
\usepackage{amsmath, amsfonts, amssymb, amsthm}
\usepackage[left=2cm,right=2cm,top=1.5cm,bottom=2cm]{geometry}
\usepackage{indentfirst}

% Plot package
\usepackage{pgfplots}
\pgfplotsset{compat=1.14}

% Psedudocode package
\usepackage{algorithm}
\usepackage{algpseudocode}
\algrenewcommand\textproc{}% Used to be \textsc
\floatname{algorithm}{Алгоритм}

\usepackage{setspace}
\onehalfspacing
\graphicspath{{pics/}}

\begin{document}
    \setlength{\abovedisplayskip}{3pt}
    \setlength{\belowdisplayskip}{3pt}

    \thispagestyle{empty}

    \begin{singlespace}
    \begin{titlepage}
        \begin{center}
            \includegraphics[height = 3cm]{msu.png}

            {\scshape Московский государственный университет имени М.~В.~Ломоносова}\\
            Факультет вычислительной математики и кибернетики\\
            Кафедра математических методов прогнозирования\\
            \centerline{\hfill\hrulefill\hrulefill\hrulefill\hrulefill\hfill}

            \vfill

            {\LARGE Отчет по бонусному заданию по МОМО: \\ Проекция на симплекс}

            \vspace{1cm}

        \end{center}

        \vfill

        \begin{flushright}
            Студент 517 группы:\\
                \textit{Оспанов А.М.}

            \vspace{5mm}

        \end{flushright}

        \vfill

        \begin{center}
            Москва, 2016
        \end{center}
    \end{titlepage}
    \end{singlespace}

    \newpage


    \section{Описание задания}
    В данном задании предлагается придумать эффективный алгоритм вычисления евклидовой проекции $\pi_{\Delta_n}(v)$
    заданной точки $v \in \mathbb{R}^n$ на симплекс:
    $$\pi_{\Delta_n}(v) = \text{arg}\min\limits_{x \in \Delta_n}||x - v||_2,$$
    где
    $$\Delta_n = \Big\{ x \in \mathbb{R}_+^n | \displaystyle\sum_{i=1}^n x_i = 1 \Big\}$$

    \section{Решение}

    Т.к. норма всегда положительна, то $\text{arg}\min\limits_{x \in \Delta_n}||x - v||_2 = \text{arg}\min\limits_{x \in \Delta_n}\frac{1}{2}||x - v||_2^2$.
    Таким образом можно решать задачу с квадратом нормы для упрощения выкладок.

    Выпишем лагранжиан:
    $$L(x; \lambda, \mu) = \dfrac{1}{2} ||x - v||_2^2 - \lambda^Tx + \mu(\displaystyle\sum_{i=1}^n x_i - 1)$$

    и систему ККТ:
    $$
    \begin{cases}
        \nabla L(x; \lambda, \mu) = 0\\
        x_i \lambda_i = 0, & i = 1, \dots, n\\
        \displaystyle\sum_{i=1}^n x_i = 1 \\
        \lambda \geq 0 \\
        x_i \geq 0 & i = 1, \dots, n\\
    \end{cases}
    $$

    Так как задача выпукла и ограничения линейные, решив эту систему, получим точку минимума.

    Посчитаем градиент лагранжиана и выразим $\lambda$:
    $$\nabla L(x; \lambda, \mu) = x - v - \lambda + \mu\vec{1} = 0$$
    $$\lambda = x - v + \mu\vec{1}$$

    Преобразуем систему ККТ подставив $\lambda$:
    $$
    \begin{cases}
        x_i(x_i - v_i + \mu) = 0\\
        x_i \geq 0\\
        \displaystyle\sum_{i=1}^n x_i = 1 \\
        x_i - v_i + \mu \geq 0\\
    \end{cases}
    $$
    где $i = 1, \dots, n$

    Теперь введем следующие множества:
    $$I_+(\mu) = \{i | x_i - v_i + \mu > 0\}$$
    $$I_0(\mu) = \{i | x_i - v_i + \mu = 0\}$$
    тогда:
    \[x_i =
    \begin{cases}
        0, & i \in I_+(\mu) \\
        v_i - \mu, & i \in I_0(\mu)\\
    \end{cases} \label{eq:x} \tag{*}
    \]

    Учитвая это, выразив $\displaystyle\sum_{i=1}^n x_i$:
    $$\displaystyle\sum_{i=1}^n x_i = \displaystyle\sum_{i \in I_+(\mu)} 0 + \displaystyle\sum_{i \in I_0(\mu)} (v_i - \mu) = 1$$
    получим:
    $$\displaystyle\sum_{i=1}^n (v_i - \mu)_+ = 1, \text{где } x_+ =
    \begin{cases}
        x, & x \geq 0 \\
        0, & x < 0 \\
    \end{cases}
    $$
    Обозначим $f(\mu) = \displaystyle\sum_{i=1}^n (v_i - \mu)_+ = \displaystyle\sum_{i=1}^n max\{v_i - \mu, 0\}$.

    Распишем свойства этой функции:
    \begin{itemize}
        \item непрерывна
        \item строго убывает ($\mu_1 < \mu_2 \Rightarrow f(\mu_1) > f(\mu_2)$)
        \item выпукла
    \end{itemize}

    Докажем эти свойства:

    1) Непрерывность: очевидна (непрерывность линейной ф-ции и $max$)

    2) Строгое убывание: Расположим $v_i$ в порядке возрастания ($v_1 \leq v_2 \leq \dots \leq v_n$). Тогда видно, что при возрастании $\mu$, ф-я убывает, причем строго.

    3) Выпуклость: $v_i - \mu$ -- линейная ф-я и следовательно выпукла. $max$ -- выуклая ф-я. $\Rightarrow$ Ф-я $f(\mu)$, как суперпозиция выпуклых ф-ций, -- выпукла.

    Далее будем считать, что $v_i$ расположены в порядке возрастания ($v_1 \leq v_2 \leq \dots \leq v_n$).

    Нарисуем график функции:

    \begin{center}
    \begin{tikzpicture}[scale=1.3]
    \begin{axis}[
        axis y line=middle,
        axis x line=middle,
        xlabel=$\mu$,
        ylabel=$f(\mu)$,
        xtick={-2, -1, 0, 1, 2, 3},
        xticklabels={$v_1$, $v_2$, $v_3$, $\dots$, $v_{n-1}$, $v_n$},
        ytick={0, 0.15, 0.35, 0.6, 1, 1.5},
        yticklabels={0, , , $\displaystyle\sum_{i=1}^n v_i [v_i > 0]$, ,},
        every axis x label/.style={
            at={(ticklabel* cs:1.05)},
            anchor=west,
        },
        every axis y label/.style={
            at={(ticklabel* cs:1.05)},
            anchor=south,
        },
        xmin=-3, xmax=4.5,
        ymin=-0.5, ymax=1.6
    ]
    \addplot coordinates {
    	(-2, 1.5) (-1, 1) (0, 0.6) (1, 0.35) (2, 0.15) (3, 0) (3.5, 0)
    };
    \addplot[mark=none, blue] coordinates {
    	(-3, 2) (-2, 1.5)
    };
    \addplot[mark=none, blue] coordinates {
    	(3, 0) (4, 0)
    };
    \end{axis}
    \end{tikzpicture}
    \end{center}

    Пусть $v_i \leq \mu \leq v_{i+1}$. Тогда $f(\mu) = \displaystyle\sum_{j=i+1}^n (v_j - \mu) = \displaystyle\sum_{j=i+1}^n v_i - (n - i)\mu = 1$

    Следовательно $\mu = \dfrac{\sum_{j=i+1}^n v_i - 1}{n - i}$

    Чтобы вычислить $x$, перепишем \eqref{eq:x}, раскрыв множества:
    \[x_i =
    \begin{cases}
        0, & \mu > v_i \\
        v_i - \mu, & \mu \leq v_i \\
    \end{cases}
    \]



    \begin{algorithm}
    \caption{Проекция на симплекс}
    \begin{algorithmic}[1]

    \Function{proj}{$v_0$}
        % \State Given: $v_0$
        \State $v = sort(v_0)$
        \For{$i = 1$ to $n - 1$}
            \State $\mu = \dfrac{\sum_{j=i+1}^n v_i - 1}{n - i}$
            \If {$\mu \in [v_i, v_{i+1}]$}
                \State $\mu_0 = \mu$
                \State \textbf{break}
            \EndIf
        \EndFor
        \For{$i = 1$ to $n - 1$}
            \If {$\mu_0 > v_{0,i}$}
                \State $x_i = 0$
            \Else
                \State $x_i = v_{0,i} - \mu_0$
            \EndIf
        \EndFor
        \State \Return $x$
    \EndFunction

    \end{algorithmic}
    \end{algorithm}

    При некоторой оптимизации (например вместо вычисления суммы в строчке 4, вычислить полную сумму вначале, а в 4й строке отнимать $v_i$),
    то алгоритм работает за $O(n + n \log n)$ времени ($O(n \log n)$ времени тратится на сортировку), что то же самое, что и $O(n \log n)$.


\end{document}
