\documentclass[12pt, a4paper]{article}
\usepackage[utf8]{inputenc}
\usepackage[russian]{babel}
\usepackage[pdftex]{graphicx, color}
\usepackage{amsmath, amsfonts, amssymb, amsthm}
\usepackage[left=2cm,right=2cm,top=1.5cm,bottom=2cm]{geometry}
\usepackage{indentfirst}
\usepackage{hyperref}

\usepackage{setspace}
\onehalfspacing
\graphicspath{{pics/}}

\begin{document}
    \begin{singlespace}
    \begin{center}
        \includegraphics[height=3cm]{msu.png}

        {\large\textbf{Домашнее задание 1 по БММО\\
        <<Сопряжённые распределения и экспоненциальный класс распределений>>}\\}

        \vspace{0.3cm}

        \textit{\textbf{Аят Оспанов}}

        617 гр., ММП, ВМК МГУ, Москва

        22 сентября 2017 г.
    \end{center}
    \end{singlespace}

    \section{Задача 1}

    Функция правдоподобия будет иметь следующий вид:

    $$p(X|\theta) = \displaystyle\prod_{i=1}^n p(x_i|\theta) =
    \displaystyle\prod_{i=1}^n \frac{1}{\theta}[x_i \leq \theta] =
    \frac{1}{\theta^n}[\max_{i=1,\dots,n}(x_i) \leq \theta]$$

    Т.к. функция правдоподобия убывает относительно $\theta$, то оценка максимального правдоподобия
    $$\theta_{ML} = \max_{i=1,\dots,n}(x_i)$$

    Как можно увидеть из вида функции правдоподобия, сопряженным распределением $p(\theta)$ будет распределение Парето:
    $$p(\theta) = Pareto(\theta|a,b) = \frac{ba^b}{\theta^{b+1}}[\theta \geq a]$$

    Теперь найдем апостериорное распределение:
    \begin{gather*}
        p(\theta|X) \propto p(X|\theta) p(\theta) =
        \frac{1}{\theta^n}[\max_{i=1,\dots,n}(x_i) \leq \theta] \cdot
        \frac{ba^b}{\theta^{b+1}}[\theta \geq a] =\\
        \frac{ba^b}{\theta^{n+b+1}}[\max_{i=1,\dots,n}(a, x_i) \leq \theta] =
        Pareto(\theta|\max(a, x_1, \dots, x_n), n+b) = Pareto(\theta|a',b')\\
        a' = \max(a, x_1, \dots, x_n)\\
        b' = n + b
    \end{gather*}

    Теперь найдем его статистики:
    \begin{gather*}
        \mathbb{E}[\theta|X] = \int\limits_{-\infty}^{+\infty}\theta Pareto(\theta|a',b') d\theta =
        \int\limits_{-\infty}^{+\infty}\theta \frac{b'a'^{b'}}{\theta^{b'+1}}[\theta \geq a']d\theta =
        b'a'^{b'} \int\limits_{a'}^{+\infty} \frac{d\theta}{\theta^{b'}} =
        b'a'^{b'} \big(0 - \frac{a'^{1-b'}}{1-b'}\big) = \\
        \frac{b'a'}{b'-1} = \frac{(n+b)\max(a, x_1, \dots, x_n)}{n+b-1}
    \end{gather*}

    \begin{gather*}
        P(\theta < \theta_{med}) = \int\limits_{-\infty}^{\theta}\frac{b'a'^{b'}}{\theta^{b'+1}}d\theta = 0.5\\
        b'a'^{b'} \int\limits_{-\infty}^{\theta} \frac{d\theta}{\theta^{b'+1}} = 0.5\\
        \Big(\frac{a'}{\theta}\Big)^{b'} = \frac{1}{2}\\
        \theta = a'\sqrt[b']{2}\\
        \theta_{med} = \sqrt[n+b]{2} \max(a, x_1, \dots, x_n)
    \end{gather*}

    Аналогично с функцией правдоподобия:
    $$mode[\theta|X] = arg\max_\theta Pareto(\theta|a',b') = a' = \max(a, x_1, \dots, x_n)$$

    \section{Задача 2}

    Пусть автобусы пронумерованы от 1 до $\theta$. Выборка автобусов берется из равномерного распределния. Тогда априорное распределение $p(\theta)$ есть распределение Парето: $p(\theta) = Pareto(\theta|1, b)$. Тут мы предполагаем, что в городе хотя бы 1 автобус: $\theta \geq 1$.

    После первого увиденного автобуса с номером 100, апостериорное распределение $p(\theta|100)$ будет распределение Парето с параметрами $a' = 100$ и $b' = b + 1$ и следующей статистикой:

    \begin{align*}
        &\mathbb{E}[\theta|x_1] = \frac{100 * (b + 1)}{b} = 100 + \frac{100}{b} \in (100, 200]\\
        &mode[\theta|x_1] = 100\\
        &\theta_{med} = 100\sqrt[b+1]{2} \in (100, 100\sqrt{2} \approx 142]
    \end{align*}

    Мода в данном случае не адекватна, т.к. считать, что автобус с номером 100 будет часто встречающимся не корректно. Матожидание адекватно при некоторых $b$, т.к. если $b = 1$, то считать, что в городе $400$ автобусов ($2\cdot$медиана) после увиденного 100-го не адекватно. Таким образом, в данном случае мода адекватна, т.к. количество автобусов окажется в интервале $(200, 283]$

    Теперь рассмотрим случай, когда мы увидели автобусы с номерами $50$ и $100$. Тогда статистика будет следующей:
    \begin{align*}
        &\mathbb{E}[\theta|x_1, x_2, x_3] = \frac{150 * (b + 3)}{b + 2} = 150 + \frac{150}{b + 2} \in (150, 200]\\
        &mode[\theta|x_1] = 150\\
        &\theta_{med} = 150\sqrt[b+1]{2} \in (150, 150\sqrt{2} \approx 213]
    \end{align*}

    В данном случае мода все еще не адекватна, т.к. мы не видели повторящиеся автобусы. Медиана и матожидание примерно одинаково адекватны, но более адекватно было бы, если бы они были равны 100.

    \section{Задача 3}

    Запишем распределение Парето при фиксированном $a$ в форме экспоненциального класса:
    $$exp(x|b) = \frac{f(x)}{g(b)}e^{bu(x)}$$

    \begin{gather*}
        Pareto(x|a,b) = \frac{ba^b}{x^{b+1}}[x \geq a] = \frac{ba^b}{x}e^{-b \log(x)}[x \geq a] =
        \frac{\frac{[x \geq a]}{x}}{\frac{1}{ba^b}} e^{b (-\log(x))}
    \end{gather*}

    В итоге получаем, что $f(x) = \frac{[x \geq a]}{x}$, $g(b) = \frac{1}{ba^b}$, $u(x) = -\log(x)$

    Найдем $\mathbb{E}\log(x)$:
    \begin{gather*}
        \mathbb{E}\log(x) = -\mathbb{E} u(x) = - \frac{d \log g(b)}{d b} = \frac{d \log(ba^b)}{d b} =
        \frac{d (\log(b) + b \log(a))}{d b} = \frac{1}{b} + \log(a)
    \end{gather*}

\end{document}
