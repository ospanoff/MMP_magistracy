\documentclass[12pt, a4paper]{article}
\usepackage[utf8]{inputenc}
\usepackage[russian]{babel}
\usepackage[pdftex]{graphicx, color}
\usepackage[left=2cm,right=2cm,top=1.5cm,bottom=2cm]{geometry}
\usepackage{indentfirst}
\usepackage{float}
\usepackage{hyperref}
\usepackage[justification=centering]{caption}
\usepackage{amsmath, amsfonts, amssymb, amsthm, amsbsy, mathtools}

\usepackage{setspace}
\onehalfspacing
\graphicspath{{pics/}}

\begin{document}
    \begin{singlespace}
    \begin{center}
        \includegraphics[height=3cm]{msu.png}

        {\large\textbf{Отчёт по первому заданию по курсу Алгоритмика\\
        <<Поиск пары ближайших точек>>}}

        \vspace{0.3cm}

        \textit{\textbf{Аят Оспанов}}

        617 гр., ММП, ВМК МГУ, Москва

        23 октября 2017 г.
    \end{center}
    \end{singlespace}

    \tableofcontents

    \section{Постановка задачи}
        Разработать и реализовать программу для нахождения в множестве Q, состоящем из $n \geq 2$ точек, пары точек, расстояние между которыми минимально среди всех пар из Q. Расстояние измеряется в манхэттенской метрике. Две точки могут совпадать, в этом случае расстояние равно нулю. Программа должна реализовывать алгоритм <<разделяй-и-властвуй>> и иметь время работы $O(n\log n)$. Исходные данные задаются в текстовом файле. Первая запись -- число точек, далее координаты точек. Координаты точек заданы действительными числами в диапазоне $[0, 10^{17}]$. Максимальное число точек $n=10^6$.

    \section{Описание метода решения}
        Был реализован метод, описанный в \cite{clrs}. с некоторой модификацией. Если по книге хранился массив $Y$ точек, отсортированных по координате $y$, и в рекурсиях подмассивы еще и сортировались, то в модификации используется массив $X$, который сортируется на этапе ``властвования'' по $y$, а на этапе ``комбинирования'' сливается. В итоге получается представленный далее алгоритм.

        \subsection{Этап разделения}
            Массив точек разбивается на два по их $x$-координатам. Для этого в отсортированном заранее массиве берется средняя точка как разделяющая.

        \subsection{Этап властвования}
            Для каждого подмассива находятся ответы $\delta_L$ и $\delta_R$. Обозначим через $\delta = \min(\delta_L, \delta_R)$. Также подмассивый сортируются по координате $y$ для этапа ``объединения''

        \subsection{Этап объединения}
            Теперь попытаемся обнаружить такие пары точек, расстояние между которыми меньше $\delta$, причём одна точка лежит в левом подмассиве, а другая -- в правом. Очевидно, что нужно рассматривать только те точки, которые отстоят от разделяющей точки меньше, чем на $\delta$. Объединим их во множество ближайших точек. И для каждой такой точки попытаемся найти точки, которые к ней ближе чем на $\delta$. Для этого достаточно рассмотреть точки, которые по координате $y$ ближе чем на $\delta$. Это можно сделать быстрее, если массив отсортирован по $y$. Тогда мы рассмотрим только те точки, которые в массиве ближайших точек лежат до данной рассматриваемой точки. Их, по доказательству из \cite{clrs}, максимум 7.

    \section{Инструкцию по работе с программой}
        Компиляция: \verb|g++ -std=c++11 main.cpp closest_points.cpp -o 2closest_pts|

        Запуск: \verb|./2closest_pts <путь до файла с данными>|.

        \quad Например \verb|./2closest_pts data/4node.txt|


    \begin{thebibliography}{9}
        \bibitem{clrs}
        Кормен Т.Х. и др. Алгоритмы: построение и анализ, 3-е изд., Москва, <<И. Д. Вильямс>>, 2016. - раздел 33.4. стр.1086-1090 -- 1328 с.
    \end{thebibliography}

\end{document}
